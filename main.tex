\documentclass[12pt]{article}
\usepackage{listings}
% You are to submit a single .pdf file with your solutions. The file should have a header
% that clearly outlines: 
% your name, 
% the course, 
% the project number, 
% and the due date.
%------------------------------------------------------------------------------------
\begin{document}
\begin{titlepage}
   \begin{center}
       \vspace*{1cm}
       \large
       \textbf{Project 1: Image Filtering and Hybrid Images}
       \normalsize

       \vspace{0.5cm}

       \textbf{Author: Gabriel Hofer}

       \vspace{0.5cm}

       CSC-414 Introduction to Computer Vision

       \vspace{0.5cm}

       Instructor: Dr. Hoover

       \vspace{0.5cm}

       Due: February 26, 2020

       \vfill

       Department: Computer Science and Engineering\\
       University: South Dakota School of Mines and Technology\\
   \end{center}
\end{titlepage}
%------------------------------------------------------------------------------------
\newpage
\section{Questions}
\textbf{Q1: Explicitly describe image convolution: the input, the transformation, and the output.
Why is it useful for computer vision?} 

Image convolution uses a kernel that is applied to each pixel in the image to create a new image.
A kernel is used to view neighboring cells in the image and calculate a new pixel value for the 
resulting image.

\textbf{Q2: What is the difference between convolution and correlation? Construct a scenario that
produces a different output between both operations and show some images of the result.
(you can use built in correlation functions and convolution functions in scipy here if
desired).}  

Convolution takes two images and produces a new third image. Convolution is also a mathematical
operation that takes to functions and creates a third function.
Correlation measures the displacement of one signal relative to the other signal. It measures 
similarity of two images.
Correlation is similar to convolution.  

\textbf{Q3: What is the difference between a high pass filter and a low pass filter in how they
are constructed and what they do to the image? Please provide example kernels and output
images.} 

The main difference between the two filters is the range of frequency that they allow to pass through them.
A high pass filter passes signals with frequencies that are above a certain threshold.
A low pass filter passes signals with frequencies that are below a certain threshold.
Low pass filters can be created by taking an images and removing or subtracting the 
high-pass signals from it.

\textbf{Q4: How does computation time vary with filter sizes from 15×15 to 3×3 (for all odd and
square sizes), and with image sizes from 0.25 MPix to 8 MPix (choose your own intervals)?
Measure both using scipy’s convolve2d to produce a matrix of values (you should
generate a 3d surface plot with image size as one axis, kernel size as the other axis, and
time as the surface). You may use the skimage.rescale function to vary the size of an
image.}

As the kernel size increases, the amount of time required to perform the convolution 
also increases. This is because the algorithm to iterate through the kernel 
has complexity $O(N*M)$ where the kernel dimensions are $N$ by $M$.

\end{document}




