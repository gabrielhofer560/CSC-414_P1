\documentclass[12pt]{article}
\usepackage{listings}
% You are to submit a single .pdf file with your solutions. The file should have a header
% that clearly outlines: 
% your name, 
% the course, 
% the project number, 
% and the due date.
%------------------------------------------------------------------------------------
\begin{document}
\begin{titlepage}
   \begin{center}
       \vspace*{1cm}
       \large
       \textbf{Project 1: Image Filtering and Hybrid Images}
       \normalsize

       \vspace{0.5cm}

       \textbf{Author: Gabriel Hofer}

       \vspace{0.5cm}

       CSC-414 Introduction to Computer Vision

       \vspace{0.5cm}

       Instructor: Dr. Hoover

       \vspace{0.5cm}

       Due: February 26, 2020

       \vfill

       Department: Computer Science and Engineering\\
       University: South Dakota School of Mines and Technology\\
   \end{center}
\end{titlepage}
%------------------------------------------------------------------------------------


\newpage
\section{Overview}
We will write an image convolution function (image filtering) and use it to create hybrid
images! The technique was invented by Oliva, Torralba, and Schyns in 2006, and published
in a paper at SIGGRAPH (available on D2L in the project1 directory tree). High frequency
image content tends to dominate perception but, at a distance, only low frequency (smooth)
content is perceived. By blending high and low frequency content, we can create a hybrid
image that is perceived differently at different distances

\section{Image Filtering}

\subsection{Task}
\textbf{Implement convolution in my imfilter(image,kernal,mode,boundary)
to imitate convolution in scipy’s convolv2d function. Your filtering algorithm should:}
\begin{itemize}
  \item Pad the input image with zeros if need be.
  \item Support grayscale and color images.
  \item Support arbitrary shaped odd-dimension kernels (e.g., 7x9 filters but not 4x5 filters).
  \item Return an error message for even-dimension filters (i.e., filters who’s dimensions are
  \item even number), as their output is undefined.
  \item Return an identical image with an identity (impulse response) kernel.
  \item Return a filtered image which is the same size and resolution as the input image.
\end{itemize}

% https://laurentperrinet.github.io/sciblog/posts/2017-09-20-the-fastest-2d-convolution-in-the-world.html



\section{Hybrid Images}

\subsection{Task 1}

\subsection{Task 2}

\subsection{Task 3}


\section{Questions}
\textbf{Q1: Explicitly describe image convolution: the input, the transformation, and the output.
Why is it useful for computer vision?}


\textbf{Q2: What is the difference between convolution and correlation? Construct a scenario that
produces a different output between both operations and show some images of the result.
(you can use built in correlation functions and convolution functions in scipy here if
desired).}

% add a citation for this -- find citation somewhere on google scholar
% https://www.mathworks.com/matlabcentral/answers/264746-difference-between-convolution-and-correlation
"Theoretically, convolution are linear operations on the signal or signal modifiers, whereas correlation is a measure of similarity between two signals."

\textbf{Q3: What is the difference between a high pass filter and a low pass filter in how they
are constructed and what they do to the image? Please provide example kernels and output
images.}

The main difference is the range of frequency that they allow to pass through it.
% https://electronicscoach.com/difference-between-high-pass-and-low-pass-filter.html
Architecture:




% 
% 
% \section{Questions}
% \small
% \textbf{1. We wish to set all pixels that have a value of 10 or less to 0, to remove camera
% sensor noise. However, our code is slow when run on a database with 1000
% grayscale images.}
% 
% \textbf{(a) How could we speed it up? Please include your code.}
% 
% From the CSC-414 Python Tutorial: 
% "Since NumPy is an extension for Python that is written in C, NumPy operations are faster than
% their corresponding Python equivalents. For example, performing matrix multiplication of two
% NumPy arrays is faster than iterating through two Python lists representing arrays and multiplying
% the correct elements. As such, we recommend doing as much of your calculations in NumPy
% as possible. It is best to avoid using for loops whenever possible; one can attain significant
% performance improvements through vectorization and logical indexing."
% 
% % --- my answer ---
% \begin{lstlisting}[frame=single,language=Python,caption=Slower For Loops with Greyscale Image\label{code:grayscale_slow}]
% import time
% import skimage
% from skimage import io
% import numpy as np
% import matplotlib.pyplot as plt
% from skimage import img_as_float32
% import sys
% 
% def process(file):
%   A = io.imread(file)
%   (m1,n1) = A.shape
%   for i in range(m1):
%     for j in range(n1):
%       if A[i,j] <= 10 :
%         A[i,j] = 0
% 
% start = time.time()
% for file in sys.argv[1:]:
%   process(file)
% end = time.time()
% print("total time: "+str(end - start))
% \end{lstlisting}
% \newpage
% \begin{lstlisting}[frame=single,language=Python,caption=Logical Indexing with Greyscale Image\label{code:grayscale_fast}]
% import time
% import skimage
% from skimage import io
% import numpy as np
% import matplotlib.pyplot as plt
% from skimage import img_as_float32
% import sys
% 
% def process(file):
%   A = io.imread(file)
%   (m1,n1) = A.shape
%   B = A < 10
%   A[B] = 0
% 
% start = time.time()
% for file in sys.argv[1:]:
%   process(file)
% end = time.time()
% print("total time: "+str(end - start))
% \end{lstlisting}
% 
% (b) What factor speedup would we receive over 1000 images? Please measure it
% and include your code
% % --- my answer ---
% 
% \begin{lstlisting}[frame=single,language=Bash,caption=Command Line Performance Testing\label{code:performance_grayscale}]
% gabe@dean:~/CSC-414_Project0$ python3 slow_gry.py dog.png 
%   grizzlypeakg.png heads.png mona_lisa.png yucca.png 
% total time: 5.245616674423218
% gabe@dean:~/CSC-414_Project0$ python3 fast_gry.py dog.png 
%   grizzlypeakg.png heads.png mona_lisa.png yucca.png 
% total time: 0.0711512565612793
% \end{lstlisting}
% \[
%   speedup = 5.245616674423218 / 0.0711512565612793
% \]
% \[
%   speedup = 73.724863452
% \]
% \newpage
% (c) Next, we wish to operate on color images. How does your speeded-up version
% from 1 (a) change for color images? Please implement and measure it, report
% the speed factor change, and include your code.
% % --- my answer ---
% 
% \begin{lstlisting}[frame=single,language=Python,caption=Slower For Loops with Color Image\label{code:color_slow}]
% import time
% import skimage
% from skimage import io
% import numpy as np
% import matplotlib.pyplot as plt
% from skimage import img_as_float32
% import sys
% 
% def process(file):
%   A = io.imread(file)
%   (m1,n1,k1) = A.shape
%   for i in range(m1):
%     for j in range(n1):
%       for k in range(k1):
%         if A[i,j,k] <= 10 :
%           A[i,j,k] = 0
% 
% start = time.time()
% for file in sys.argv[1:]:
%   process(file)
% end = time.time()
% print("total time: "+str(end - start))
% \end{lstlisting}
% \newpage
% \begin{lstlisting}[frame=single,language=Python,caption=Logical Indexing with Color Image\label{code:color_fast}]
% import time
% import skimage
% from skimage import io
% import numpy as np
% import matplotlib.pyplot as plt
% from skimage import img_as_float32
% import sys
% 
% def process(file):
%   A = io.imread(file)
%   (m1,n1,k1) = A.shape
%   B = A < 10
%   A[B] = 0
%   
% start = time.time()
% for file in sys.argv[1:]:
%   process(file)
% end = time.time()
% print("total time: "+str(end - start))
% \end{lstlisting}
% 
% \begin{lstlisting}[frame=single,language=Bash,caption=Command Line Performance Testing\label{code:performance_grayscale}]
% gabe@dean:~/CSC-414_Project0$ python3 slow_rgb.py grizzlypeak.jpg 
%   total time: 11.221831321716309
% gabe@dean:~/CSC-414_Project0$ python3 fast_rgb.py grizzlypeak.jpg 
% total time: 0.08472013473510742
%   gabe@dean:~/CSC-414_Project0$ 
% \end{lstlisting}
% \[
%   speedup = 11.221831321716309 / 0.08472013473510742
% \]
% \[
%   speedup = 132.457666135
% \]
% We conclude that Color images get more speedup than Grayscale images since 132 is greater than 72.
% 
% \newpage
% \textbf{2. Suppose we wish to reduce the brightness of an image by editing the values in its
% matrix. But, when trying to visualize the result, we see some “errors”.}
% 
% \textbf{(a) What is incorrect with this approach? How can it be fixed while maintaining
% the same intended brightness reduction? Please include your code and result
% image.}
% % --- my answer ---
% 
% Errors occur because pixel values have to be in the range [0,255]. When 50
% is subtracted from pixel values less than 50, the result is a negative value, 
% which is an error. To prevent this from happening, two separate operations should
% be performed: 
%   1) pixels with a value that is less than 50 should be set to zero
%   2) 50 should be subtracted from pixels with a value that is greater than or equal to 50.

\end{document}




